\section{Lezione 1} \hfill 24/02/2026
%\addcontentsline{toc}{section}{Lezione 1}
\subsection{Cos'è un bosco?}
Non esiste un'unica definizione di bosco, ma cambia a seconda del soggetto interessato:
\begin{enumerate}
    \item per la legge ci sono alcuni vincoli numerici, dovendoci essere la certezza del diritto (es. 3000$\text{m}^2$, almeno $30\%$ di copertura,...) \textrightarrow{} sorge il dubbio su quale riferimento prendere (copertura delle chiome? Da tronco a tronco?);
    \item per il geografo può essere la ``superficie ricoperta da alberi'';
    \item per l'ecologo può essere ``l'ecosistema forestale nel quale l'ambiente è fortemente influenzata dalla presenza di alberi che cambiano la luce, il clima, la vegetazione,... rispetto all'esterno'';
    \item per il selvicoltore può essere ``l'ecosistema forestale in grado di cambiare il clima interno rispetto all'esterno (non c'è quindi un vincolo dimensionale di legge), in grado di fornire servizi o prodotti (ovvero svolgere una funzione)'';
\end{enumerate}
I ``servizi ecosistemici'' sono una PUTTANATA, creata da un economista per valutare il valore economico di un bosco. Per un forestali i servizi ecosistemici sono legati alla quantità e qualità dei servizi e dei beni che un bosco restituisce.
\subsection{Le funzioni di un bosco}
Il bosco, di per sè, non possiede nessuna funzione; è l'uomo che ne attribuisce un certo valore.\\
Esistono diverse funzioni del bosco:
\begin{multicols}{2}
\begin{enumerate}
    \item funzione protettiva;
    \item funzione faunistica;
    \item funzione ricreativa;
    \item funzione produttiva;
    \item funzione paesaggistica/paesistica;
    \item funzione turistica;
    \item funzione climatica;
    \item funzione scientifica;
    \item funzione ecologica;
    \item funzione storico/culturale;
    \item funzione sanitaria.
\end{enumerate}
\end{multicols}
\subsubsection{Funzione produttiva}
Funzione più antica ed importante per l'uomo.\\
È divisibile in due gruppi:
\begin{enumerate}
    \item produzione legnosa (tendenzialmente il legno e legname degli alberi);
    \item produzione non legnosa, ovvero che può derivare dagli alberi (es. resina, gemma, manna, frutti) o dall'ecosistema del bosco (es. funghi, selvaggina, frutti di bosco, felci);
    \item la cellulosa può essere considerata a parte, a metà via tra la produzione legnosa e non legnosa.
\end{enumerate}
\subsubsection{Funzione protettiva}
È la seconda funzione più antica del bosco, come protezione di un bene.\\
La protezione è divisibile in due categorie:
\begin{enumerate}
    \item protezione diretta: riguarda la tutela di un certo bene (es. la ferrovia o la pista da sci a valle, impedendo il movimento dei massi).\\
    Le prime leggi sulla gestione dei boschi (1200-1300), riguardano i popolamenti da protezione diretta (chiamati anche boschi sacri o boschi banditi).\\
    Gli Svizzeri hanno dato un valore economico ai boschi banditi di protezione delle infrastrutture ferroviarie, equiparandoli ai valori dei beni trasportati in un anno in quella data linea.
    \item protezione indiretta: è quella rivolta soprattutto al suolo, che qualunque bosco possiede (permette la vita umana sulla Terra). Avviene l'attenuazione dell'effetto della pioggia, riducendo l'energia cinetica e l'erosione, prevenendo così l'avvenimento di frane superficiali.\\
    Per questo motivo esistono le leggi forestali: la protezione indiretta attribuisce al bosco una funzione pubblica (oltre a quella di utilizzazione privatista già presente). 
\end{enumerate}
\subsubsection{Funzione turistico-ricreativa}
Questa funzione viene valutata con una visione ``da dentro''.\\
Il bosco è frequentato dai turisti (non sempre, poichè le persone solitamente hanno paura del bosco). Il bosco, affinchè possa svolgere la funzione turistica, dev'essere preparato, possedendo per esempio sentieri, accessori (es. panchine), devono esserci giochi di luci delle chiome (non avendo quindi una copertura completa).\\
Il bosco è ricreativo quando viene frequentato per lo svolgimento di vari sport. 
\subsubsection{Funzione paesaggistica}
Questa funzione viene valutata con una visione ``da fuori''.\\
Il paesaggio viene apprezzato maggiormente se è composto da diverse patches (disomogeneità).
\subsubsection{Funzione sanitaria}
È quella maggiormente complessa da valutare.\\
Riguarda i benefici, umani ed ambientali, che il bosco permette di ottenere (come per esempio la depurazione dell'aria e dell'acqua).\\
Mentre risulta relativamente facile valutare i benefici fisici prodotti, la valutazione oggettiva dei benefici mentali è difficile.
\subsubsection{La/le funzioni ecologiche}
La funzione ecologica include sia la concezione ecosistemica che di biodiversitaria.\\
Il bosco è il sito con maggior biodiversità tra tutti gli ecosistemi (maggior concentrazione).\\
I boschi non sono inamovibili, ma seguono i naturali cicli terrestri (nascita, crescita, senescenza e morte). Inoltre, le condizioni ambientali (che sia per motivi naturali o antropici) cambiano: per assurdo, per mantenere un ecosistema naturale, occorre applicare scelte innaturali (artificiali). 
\subsubsection{Funzione museale}
Inteso come testimonianza dell'ambiente.\\
Gli alberi monumentali ``hanno avuto un culo della madonna'', poichè non sono stati soggetti a disturbi mortali per tutto il periodo della loro vita.
\subsubsection{Funzione di sink di carbonio}
È un concetto relativamente limitante, poichè dipende dalle condizioni del bosco. Se il popolamento cresce, avviene lo stoccaggio del carbonio; un bosco in stato di equilibrio invece ha alti consumi, che portano lo stoccaggio a zero.
\\[10pt]
Generalmente, il selvicoltore riesce a gestire il bosco tenendo in considerazione solamente 1-2 funzioni del bosco, rendendo le altre accessorie. Risulta quindi importante conoscere quale funzione principale viene attribuita al bosco.
%\subsection{Differenze tra bosco e foresta}
\paragraph{Differenze tra bosco e foresta}
I termini bosco e foresta solitamente sono intesi come sinonimi.\\
Il bosco è considerato come l'ecosistema arboreo, mentre la foresta è l'ecosistema forestale (ovvero l'insieme tra il bosco, la componente idrica, le eventuali radure,...).
\paragraph{Gestione forestale e necromassa}
Quando si svolge la gestione (produttiva) di un bosco è importante considerare solamente la parte viva del soprassuolo. La necromassa non ricopre il ruolo di produzione.\\
La necromassa non è gestibile e ha la tendenza a sparire (marcendo).\\
Es. la gestione del bosco per la produzione di tavolame non porta a considerare la necromassa; invece, andrà tenuta di conto quando si vuole considerare l'aspetto di biodiversità di quel popolamento. Infatti, il legno a terra non ha un fine produttivo.\\
Una gestione oculata del bosco considera tutte le funzioni del bosco (produttiva, di biodiversità, ricreativa,...).