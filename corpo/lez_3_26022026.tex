\section{Lezione 3} \hfill 26/02/2026\\
Ogni specie arborea ha un proprio optimum ecologico: ne deriva che non esiste una singola selvicoltura, poichè il comportamento dell'albero può cambiare in base alle condizioni stazionali. Per questo motivo non esiste una selvicoltura per l'abete rosso, per il faggio,..., bensì esiste una selvicoltura per una faggeta, una pecceta...\\
Ogni specie arborea potrebbe essere presente in boschi diversi, con problematiche diverse.\\
Risulta necessario provare ad identificare, mediante diversi parametri, i diversi popolamenti forestali.
\paragraph{Tipologie forestali} Non si parla di specie, ma di bosco.\\
Ci sono due motivi per cui sono state inventate le tipologie forestali:
\begin{enumerate}
    \item i botanici si basano sulle specie presenti: le associazioni possono essere una delle fasi del bosco (es. quando il bosco nasce ha una composizione diversa rispetto a quella del popolamento maturo);
    \item sono state introdotte specie diverse (es. il castagneto di per sè non esisterebbe, poichè l'uomo ha introdotto il castagno in un popolamento di querce).
\end{enumerate}
I tipi nascono in Svizzera e sono stati successivamente creati altri per il contesto italiano. Le tipologie italiane sono organizzate a livello regionale, poichè sono diverse le condizioni climatico-ambientali del territorio.
\paragraph{Gestione} Gli interventi selvicolturali possono essere diversi a seconda delle condizioni stazionali differenti.
\subsection{Classificazione}
La classificazione si basa su quattro livelli:
\begin{enumerate}
    \item categoria: indica normalmente la composizione (es. pineta, pecceta, querceto, castagneto,...);
    \item tipo: associa alla categoria uno o più caratteristiche che lo inquadrano al meglio;
    \item sottotipo: non prevede un cambiamento nella gestione che si fa;
    \item variante: tipo che ha una caratteristica inusuale (es. presenza di specie rara, caratteristica da tenere conto nella selvicoltura).
\end{enumerate}
Per semplicità si considerano solamente le categorie ed i tipi.
\subsubsection{Categorie e tipi}
\paragraph{Faggeta tipica} Indica la faggeta nell'optimum.\\
\subsubsection{Regione forestale}
Una seconda descrizione deriva dalla regione forestale; è un'area che può cambiare di superficie e dimensione a seconda delle caratteristiche climatiche (principalmente precipitazione).
\begin{enumerate}
    \item Faggeta costiera: la regione costiera è influenzata dal clima mediterraneo:
    \begin{itemize}
        \item estate secca ed inverno umido;
        \item la temperature sono torride d'estate e temperate d'inverno;
        \item le specie che colonizzano questa regione devono essere adattate a questo clima (soprattutto per la siccità estiva).
    \end{itemize}
    \item Faggeta planiziale: in Italia è relativo alla pianura padana. È influenzato dalle correnti costiere, non essendoci barriere:
    \begin{itemize}
        \item il clima è sub-mediterraneo;
        \item i massimi di precipitazione sono in primavera ed in autunno (tendenzialmente di più in autunno, data la maggiore quantità di energia in aria);
        \item due periodi secchi: uno estivo (in luglio) ed uno in inverno (gennaio);
        \item grandi perturbazioni, sia estive che invernali: verso fine giugno, verso fine luglio e verso fine agosto, prima settimana di dicembre, tra Natale e Capodanno e verso fine gennaio;
        \item in inverno solitamente c'è neve;
        \item le temperature possono essere molto calde d'estate e molto fredde d'inverno (inversione termica)
    \end{itemize}
    \item Faggeta avanalpica-collinare: area ``prima'' delle montagne (es. per il Veneto si possono vedere i Colli Berici, i Colli Euganei, le Colline del Prosecco):
    \begin{itemize}
        \item c'è una leggera elevazione di quota rispetto alla pianura;
        \item il regime delle precipitazione rimane uguale alla pianura, ovvero primaverile-autunnale, ma piove leggermete di più (1000 mm annui);
        \item non c'è più l'inversione termica: rispetto alla pianura le minime sono maggiori e le massime sono minori (non ci sono gli eccessi);
        \item il clima è favorevole a molte specie.
    \end{itemize}
    \item Faggeta esalpica: sono individuabili come le prime montagne:
    \begin{itemize}
        \item le correnti, incontrando le prime montagne, scaricano i loro contenuti di acqua: grandi quantità di piogge, fino a 3000 mm annui (es. Friuli Venezia Giulia);
        \item il clima è più freddo, con una media annua di 8 gradi centigradi.
    \end{itemize}
    \item Faggeta mesalpica: quando le correnti passano le prime montagne, hanno perso molta acqua:
    \begin{itemize}
        \item le precipitazioni tornano a diminuire (800-1000 mm annui);
        \item le temperature tornano ad abbassarsi.
    \end{itemize}
    \item Faggeta endalpica
    \item Faggeta appenninica 
\end{enumerate}
Le faggete costiere e planiziali non esistono.